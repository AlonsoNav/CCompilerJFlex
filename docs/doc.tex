\documentclass[a4paper,12pt]{article}

\usepackage{graphicx}   % Imágenes
\usepackage{helvet}     % Fuente de letra
\usepackage[hidelinks]{hyperref}   % Tabla de contenido
\usepackage{url}
\usepackage{etoolbox}
\usepackage{setspace}   % Interlineado

% Eliminar título de la bibliografía
\patchcmd{\thebibliography}{\section*{\refname}}{}{}{}

\onehalfspacing

\renewcommand{\familydefault}{\sfdefault}
\renewcommand{\contentsname}{Tabla de contenidos}

% Configuración de la portada
\begin{document}

\begin{titlepage}
    \centering
    \vspace*{0.5cm}

    % Logo de la universidad
    \includegraphics[width=1\textwidth]{logo-tec.png}\par\vspace{1cm}

    % Nombre de la universidad
    {\scshape Instituto Tecnológico de Costa Rica\par}
    \vspace{2cm}

    % Título
    {\Huge\bfseries Proyecto \#1: Scanner\par}
    \vspace{2cm}

    % Escuela
    {\large Escuela de Ingeniería en Computación\par}

    % Curso
    {\large Compiladores e Intérpretes IC-5701\par}
    \vspace{2cm}

    % Autor
    {\large Alonso Navarro Carrillo, c. 2022236435\par}
    \vspace{0.25cm}
    {\large Carlos, c. \par}
    \vspace{0.25cm}
    {\large Valeria, c. \par}
    \vspace{2cm}

    \vfill

    % Profesor
    {\large Ing. Ericka Marín Schumann\par}

    % Semestre
    {II Semestre 2024\par}
\end{titlepage}

\tableofcontents\newpage

% Introducción
\section*{Introducción}
\addcontentsline{toc}{section}{Introducción}

% Estrategia de solución
\section*{Estrategia de solución}
\addcontentsline{toc}{section}{Estrategia de solución}
\begin{flushleft}
    \hspace*{2em} Después de leer extensamente la documentación 
    de JFlex, se comenzó a diseñar las expresiones regulares 
    de los tokens que el scanner debía reconocer. Aquí surgió 
    el primer problema: el scanner reconoce tokens según la 
    prioridad de orden. Es decir, si la primera expresión 
    regular es un punto, no se reconocerá ningún otro token, 
    ya que este metacaracter coincidiría con cualquier 
    carácter, identificándolo como un error.
\end{flushleft}

% Análisis de resultados
\section*{Análisis de resultados}
\addcontentsline{toc}{section}{Análisis de resultados}

% Lecciones aprendidas
\section*{Lecciones aprendidas}
\addcontentsline{toc}{section}{Lecciones aprendidas}

% Casos de prueba
\section*{Casos de prueba}
\addcontentsline{toc}{section}{Casos de prueba}

% Manual de usuario
\section*{Manual de usuario}
\addcontentsline{toc}{section}{Manual de usuario}

% Bitácora
\section*{Bitácora}
\addcontentsline{toc}{section}{Bitácora}

\subsection*{Fecha: 26-08-2024}
\begin{flushleft}
    \hspace*{2em} En la primera reunión del equipo de trabajo,
    se acordó que CV se encargará de los expresiones regulares
    de los operadores y del formato de impresión de la tabla.
    AN diseñará la estructura de los tokens y sus errores, 
    así como las expresiones regulares de los identificadores
    y palabras reservadas. VG se responsabilizará de los 
    literales. Además, se decidió que la documentación se 
    realizará en LaTeX y que GitHub será utilizado como 
    sistema de control de versiones.
\end{flushleft}

% Bibliografía
\newpage
\section*{Bibliografía}
\addcontentsline{toc}{section}{Bibliografía}
\begin{thebibliography}{9}

\bibitem{examplewebsite}
Klein, G., Rowe, S., \& Décamps, R. (marzo de 2023).
\emph{JFlex User’s Manual}.
JFlex Team.
En: \url{https://www.jflex.de/manual.html}.

\end{thebibliography}

\end{document}