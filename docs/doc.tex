\documentclass[a4paper,12pt]{article}

\usepackage{graphicx}   % Imágenes
\usepackage{helvet}     % Fuente de letra
\usepackage[hidelinks]{hyperref}   % Tabla de contenido
\usepackage{url}
\usepackage{etoolbox}
\usepackage{setspace}   % Interlineado
\usepackage{geometry}   % Márgenes
\usepackage{tabularx}   % Tablas

% Márgenes
\geometry{
    a4paper,
    left=25mm,
    right=25mm,
    top=25mm,
    bottom=25mm
}

% Eliminar título de la bibliografía
\patchcmd{\thebibliography}{\section*{\refname}}{}{}{}

\onehalfspacing

\renewcommand{\familydefault}{\sfdefault}
\renewcommand{\contentsname}{Tabla de contenidos}

% Configuración de la portada
\begin{document}

\begin{titlepage}
    \centering
    \vspace*{0.5cm}

    % Logo de la universidad
    \includegraphics[width=1\textwidth]{logo-tec.png}\par\vspace{1cm}

    % Nombre de la universidad
    {\scshape Instituto Tecnológico de Costa Rica\par}
    \vspace{2cm}

    % Título
    {\Huge\bfseries Proyecto \#2: Parser\par}
    \vspace{2cm}

    % Escuela
    {\large Escuela de Ingeniería en Computación\par}

    % Curso
    {\large Compiladores e Intérpretes IC-5701\par}
    \vspace{2cm}

    % Autor
    {\large Alonso Navarro Carrillo, c. 2022236435\par}
    \vspace{0.25cm}
    {\large Carlos Venegas Masis, c. 2022153870 \par}
    \vspace{0.25cm}
    {\large Valeria Gómez Acuña, c. 2022173229 \par}
    \vspace{2cm}

    \vfill

    % Profesor
    {\large Ing. Ericka Marín Schumann\par}

    % Semestre
    {II Semestre 2024\par}
\end{titlepage}

\tableofcontents\newpage

% Introducción
\section*{Introducción}
\addcontentsline{toc}{section}{Introducción}
\begin{flushleft}
	\hspace*{2em} Este proyecto se ubica en la segunda etapa de la creación de un compilador para el lenguaje de programación C, conocida como el Parsing o Análisis Sintáctico. El propósito principal de esta etapa es diseñar y desarrollar un parser que logre identificar como se relacionan los tokens que se obtenidos de la fase de análisis léxico y determinar si estos tokens forman expresiones, sentencias o estructuras válidas para un programa escrito en lenguaje C. Para lograr esto, se utilizaron dos herramientas Java CUP y JFlex, Java CUP permitió definir reglas gramaticales que describen como deben de combinarse los tokens para ser reconocidos por el parser como correctos. Mientras que JFlex fue utilizado en la etapa de análisis Léxico, esta estapa se desarrolló en la primera fase del proyecto. \par
\vspace{1em}
\hspace*{2em} A lo largo de este documento se detalla el proceso de desarrollo 
del scanner, desde la planificación hasta la implementación y evaluación de los
resultados obtenidos. Se describen decisiones técnicas tomadas para garantizar el
correcto funcionamiento del proyecto, así como las estrategias empleadas para la 
solución de los problemas encontrados durante su desarrollo. 

\end{flushleft}

% Estrategia de solución
\section*{Estrategia de solución}
\addcontentsline{toc}{section}{Estrategia de solución}
\begin{flushleft}

\hspace*{2em} Después de leer detenidamente la documentación 
    de Java CUP, se comenzó a planear y diseñar la manera en que se iba a conectar el Analizador léxico con el Analizador Sintáctico, 
    aquí surguió el primer problema porque para que funcionaran juntos era necesario modificar la estructura del Analizador Léxico, 
    de manera que devolviera al Analizador Sintáctico los tokens ya analizados. Una vez que se logró hacer esta conexión correctamente, 
    se comenzó con el diseño de las producciones, para ello primero se definieron los terminales y no terminales, luego se inició con el diseño de las 
    producciones de cada uno de los no terminales, aquí surgió el segundo problema porque algunas producciones no eran lo suficientemente 
    específicas en su composición, generando errores debido a ambiguedades, para resolver estos problemas el grupo se reunió y se definieron 
    nuevos no terminales que ayudaron a eliminar ambiguedades y también con la factorización de algunas producciones. \par
   \vspace{1em}
    \hspace*{2em} Cuando se tuvieron las producciones listas se hicieron pruebas para verificar que funcionaban correctamente, 
    luego se comenzó con el manejo de errores para esto se implementó un método llamado Syntax Error este método permite 
    manejar los errores de sintaxis para que el parser los detecte y reporte, los errores se definieron como producciones de los 
   no terminales. POr último se realizaron pruebas para verificar que el parser reconoce las reglas gramaticales e identifica todos los errores. \par
    \vspace{1em}
    
\end{flushleft}

\newpage

% Análisis de resultados
\section*{Análisis de resultados}
\addcontentsline{toc}{section}{Análisis de resultados}
\begin{table}[!ht]
    \centering
    \begin{tabularx}{\textwidth}{|X|X|X|}
        \hline
        Actividad & Porcentaje realizado & Justificación \\ 
        \hline
        Desplegar lista de errores léxicos & 100\% & \\
        \hline
        Desplegar lista de errores sintácticos & 100\% & \\
        \hline
        Evitar desplegar errores sintácticos en cáscada & 50\% & En situaciones muy específicas el parser se puede caer o dar errores poco precisos. Esto debido a complicaciones en la gramática. \\
        \hline
        Implementar las funciones de read and write & 100\% & \\
        \hline
        Declaración de variables, constantes y lista de variables & 100\% & \\
        \hline
        Reconocer la estructura del programa & 100\% & \\
        \hline
        Identificar funciones & 100\% & \\
        \hline
        Analizar expresiones con operadores aritméticos o booleanos & 100\% & \\
        \hline
        Identificar todas las estructuras de control & 100\% & \\
        \hline
        Definir buenos mensajes de error & 80\% & La mayoría de errores están bien definidos, pero si el flujo es erróneo no se obtiene un buen mensaje. \\
        \hline
    \end{tabularx}
\end{table}

% Lecciones aprendidas
\section*{Lecciones aprendidas}
\addcontentsline{toc}{section}{Lecciones aprendidas}
\begin{flushleft}
	\hspace*{2em} Durante el desarrollo del proyecto, el grupo 
pudo observar que en el parser es fundamental definir correctamente 
la gramática y las reglas de producción. La estructura de las reglas 
y el uso adecuado de precedencia y asociatividad permitieron reducir 
conflictos de análisis, como conflictos shift/reduce o reduce/reduce, 
que dificultan el proceso de análisis sintáctico y provocan errores 
de interpretación. Además, organizar adecuadamente las producciones y 
sus alternativas evitó que ciertas estructuras se interpretaran de 
forma ambigua, lo cual agregó robustez al parser al generar una mejor 
comprensión de la jerarquía entre expresiones y bloques de código. \par
\vspace{1em}
\hspace*{2em} Definir mensajes de error claros fue un reto, especialmente 
al manejar errores sintácticos complejos. En muchos casos, el parser debía 
identificar la causa específica de un error (como la falta de un punto y 
coma o el uso incorrecto de una palabra reservada) y generar un mensaje 
informativo que ayudara a identificar el problema en la línea correspondiente. 
Esto fue esencial para mejorar la experiencia del usuario al facilitar la 
corrección de errores en el código fuente. Asimismo, el equipo experimentó 
con la recuperación de errores para continuar el análisis a pesar de 
errores sintácticos, lo cual permitió al parser brindar un reporte de 
múltiples errores en una sola ejecución, aumentando la eficiencia del 
proceso de depuración.\par
\vspace{1em}
\hspace*{2em} Al enfrentarse a estructuras sintácticas complejas como 
anidación de bloques y condiciones, el equipo comprendió la importancia 
de anticipar casos especiales en las reglas de producción. Definir reglas 
claras para estos casos evitó ambigüedades en el análisis y permitió un 
procesamiento más preciso del código fuente. Esto también permitió al parser 
identificar y manejar constructos incompletos o mal formados de manera 
más efectiva, brindando así una base sólida para la interpretación y 
compilación del código.\par
\vspace{1em}

\end{flushleft}

% Casos de prueba
\section*{Casos de prueba}
\addcontentsline{toc}{section}{Casos de prueba}
\subsection*{Caso de prueba 1: Errores en Declaración, Asignación y Estructuras de Control}
\addcontentsline{toc}{subsection}{Casos de prueba 1: Errores en Declaración, Asignación y Estructuras de Control}
\begin{flushleft}
    \begin{verbatim}
    void funcion(){
    	int a = 5;
    	int b = 10
    	a + b = c; // Error: asignación inválida
    	if (a < b) {
    		print(a);
    	} else {
    		print(b)
    	}
    	while (a < b) {
    		a = a + 1;
    	}
    	for (int i = 0; i < 10; i++) {
    		print(i);
    	}
    	a + (b * 2; // Error: paréntesis no cerrado
    }
    \end{verbatim}
    Errores esperados:
    \begin{itemize}
        \item Error de sintaxis en línea 3: Falta punto y coma.
        \item Error de sintaxis en línea 4: Error antes de '=' en asignación. 
        \item Error de sintaxis en línea 8: Falta punto y coma.
    	\item Error de sintaxis en línea 16: Error en paréntesis de expresión.
    \end{itemize}
    Resultados:
    \begin{verbatim}
Starting parsing process...
Error de sintaxis en línea 3: Falta punto y coma.
Error de sintaxis en línea 4: Error antes de '=' en asignación. 
Error de sintaxis en línea 8: Falta punto y coma.
Error de sintaxis en línea 16: Error en paréntesis de expresión.
    \end{verbatim}
\end{flushleft}

\subsection*{Caso de prueba 2: Errores de Paréntesis, Puntos y Comas, y Variables No Declaradas}
\addcontentsline{toc}{subsection}{Casos de prueba 2: Errores de Paréntesis, Puntos y Comas, y Variables No Declaradas}
\begin{flushleft}
    \begin{verbatim}
    void otraFuncion() {
    	int x = 3;
    	int y = 7;
    	
    	if (x > y {
    		print(x);
    	} else {
    		print(y;
    	}
    	
    	for ( int j = 0; j < 5; j++ ) {
    		print(j)
    	}
    	while (x < y {
    		x = x + 1;
    	}
    	x = x + y // Error: falta punto y coma
    	z = x * 2; // Error: variable no declarada
    }
    \end{verbatim}
Errores esperados:
\begin{itemize}
    	\item Error de sintaxis en línea 8: Falta paréntesis de cierre en la llamada de función.
    	\item Error de sintaxis en línea 7: Error en el paréntesis de cierre del if.
    	\item Error de sintaxis en línea 12: Falta punto y coma.
    	\item Error de sintaxis en línea 16: Error en el paréntesis de cierre del while.
    	\item Error de sintaxis en línea 17: Falta punto y coma.
\end{itemize}
Resultados:
\begin{verbatim}
Starting parsing process...
Error de sintaxis en línea 8: Falta paréntesis de cierre en la llamada de función.
Error de sintaxis en línea 7: Error en el paréntesis de cierre del if.
Error de sintaxis en línea 12: Falta punto y coma.
Error de sintaxis en línea 16: Error en el paréntesis de cierre del while.
Error de sintaxis en línea 17: Falta punto y coma.

\end{verbatim}
\end{flushleft}

\newpage

\subsection*{Caso de prueba 3: Identificadores}
\addcontentsline{toc}{subsection}{Caso de prueba 3: Identificadores}
\begin{flushleft}
	\begin{verbatim}
		#include <stdio.h>
		
		
		int 1invalidVar = 10; // Identificador comienza con un dígito
		int validVar = 20; // Identificador válido
		float 2anotherInvalid = 30.5; // Identificador comienza con un dígito
		char special#CharVar = 'A'; // Identificador contiene un carácter especial '#'
		int valid_var_123 = 50; // Identificador válido
		
		
		void function() {
			int invalid@Var = 100; // Identificador contiene un carácter especial '@'
			int valid_var = 0; // Identificador válido
			float invalid%percent = 40.2; // Identificador contiene un carácter especial '%'
			double validDoubleVar = 123.456; // Identificador válido
			double 3invalidStart = 987.654; // Identificador comienza con un dígito
		}
		
		
		int main() {
			printf("Valor de validVar: %d\n", validVar); // Identificador válido
			printf("Valor de valid_var_123: %d\n", valid_var_123); // Identificador válido
			
			return 0;
		}
	\end{verbatim}
	Errores esperados:
	\begin{itemize}
		\item Error en la línea 1: Operador inválido '\#'.
		\item Error en la línea 3: Dígito antes de id '1invalidVar'.
		\item Error en la línea 3: Dígito antes de id '1invalidVar'.
		\item Error en la línea 5: Dígito antes de id '2anotherInvalid'.
		\item Error en la línea 6: Operador inválido '\#'.
		\item Error en la línea 10: Operador inválido '@'.
		\item Error en la línea 14: Dígito antes de id '3invalidStart'.
	\end{itemize}
	
	Resultados:
	\begin{verbatim}
		Errors: 
		Character unknown: # in 1
		Digit before id: 1invalidVar in 3
		Digit before id: 2anotherInvalid in 5
		Character unknown: # in 6
		Character unknown: @ in 10
		Digit before id: 3invalidStart in 14
		+------------+-----------------+------------------------------------------+
		| Token      | Tipo de Token   | Linea                                    |
		+------------+-----------------+------------------------------------------+
		| char       | KEYWORD         | 6                                        |
		+------------+-----------------+------------------------------------------+
		| double     | KEYWORD         | 13, 14                                   |
		+------------+-----------------+------------------------------------------+
		| float      | KEYWORD         | 5, 12                                    |
		+------------+-----------------+------------------------------------------+
		| int        | KEYWORD         | 3, 4, 7, 10, 11, 17                      |
		+------------+-----------------+------------------------------------------+
		| return     | KEYWORD         | 21                                       |
		+------------+-----------------+------------------------------------------+
		| void       | KEYWORD         | 9                                        |
		+------------+-----------------+------------------------------------------+
		| CharVar    | ID              | 6                                        |
		+------------+-----------------+------------------------------------------+
		| Var        | ID              | 10                                       |
		+------------+-----------------+------------------------------------------+
		| function   | ID              | 9                                        |
		+------------+-----------------+------------------------------------------+
		| h          | ID              | 1                                        |
		+------------+-----------------+------------------------------------------+
		| include    | ID              | 1                                        |
		+------------+-----------------+------------------------------------------+
		| invalid    | ID              | 10, 12                                   |
		+------------+-----------------+------------------------------------------+
		| main       | ID              | 17                                       |
		+------------+-----------------+------------------------------------------+
		| percent    | ID              | 12                                       |
		+------------+-----------------+------------------------------------------+
		| printf     | ID              | 18, 19                                   |
		+------------+-----------------+------------------------------------------+
		| special    | ID              | 6                                        |
		+------------+-----------------+------------------------------------------+
		| stdio      | ID              | 1                                        |
		+------------+-----------------+------------------------------------------+
		| validDoubl | ID              | 13                                       |
		| eVar       |                 |                                          |
		+------------+-----------------+------------------------------------------+
		| validVar   | ID              | 4, 18                                    |
		+------------+-----------------+------------------------------------------+
		| valid_var  | ID              | 11                                       |
		+------------+-----------------+------------------------------------------+
		| valid_var_ | ID              | 7, 19                                    |
		| 123        |                 |                                          |
		+------------+-----------------+------------------------------------------+
		| (          | OPERATOR        | 9, 17, 18, 19                            |
		+------------+-----------------+------------------------------------------+
		| )          | OPERATOR        | 9, 17, 18, 19                            |
		+------------+-----------------+------------------------------------------+
		| ,          | OPERATOR        | 18, 19                                   |
		+------------+-----------------+------------------------------------------+
		| .          | OPERATOR        | 1                                        |
		+------------+-----------------+------------------------------------------+
		| ;          | OPERATOR        | 3, 4, 5, 6, 7, 10, 11, 12, 13, 14, 18, 1 |
		|            |                 | 9, 21                                    |
		+------------+-----------------+------------------------------------------+
		| =          | OPERATOR        | 3, 4, 5, 6, 7, 10, 11, 12, 13, 14        |
		+------------+-----------------+------------------------------------------+
		| {          | OPERATOR        | 9, 17                                    |
		+------------+-----------------+------------------------------------------+
		| }          | OPERATOR        | 15, 22                                   |
		+------------+-----------------+------------------------------------------+
		| %          | OPERATOR_ARITHM | 12                                       |
		|            | ETIC            |                                          |
		+------------+-----------------+------------------------------------------+
		| <          | OPERATOR_RELATI | 1                                        |
		|            | ONAL            |                                          |
		+------------+-----------------+------------------------------------------+
		| >          | OPERATOR_RELATI | 1                                        |
		|            | ONAL            |                                          |
		+------------+-----------------+------------------------------------------+
		| 0          | LITERAL_INT     | 11, 21                                   |
		+------------+-----------------+------------------------------------------+
		| 10         | LITERAL_INT     | 3                                        |
		+------------+-----------------+------------------------------------------+
		| 100        | LITERAL_INT     | 10                                       |
		+------------+-----------------+------------------------------------------+
		| 20         | LITERAL_INT     | 4                                        |
		+------------+-----------------+------------------------------------------+
		| 50         | LITERAL_INT     | 7                                        |
		+------------+-----------------+------------------------------------------+
		| "Valor de  | LITERAL_STR     | 18                                       |
		| validVar:  |                 |                                          |
		| %d\n"      |                 |                                          |
		+------------+-----------------+------------------------------------------+
		| "Valor de  | LITERAL_STR     | 19                                       |
		| valid_var_ |                 |                                          |
		| 123: %d\n" |                 |                                          |
		+------------+-----------------+------------------------------------------+
		| 'A'        | LITERAL_CHAR    | 6                                        |
		+------------+-----------------+------------------------------------------+
		| 123.456    | LITERAL_DOUBLE  | 13                                       |
		+------------+-----------------+------------------------------------------+
		| 30.5       | LITERAL_DOUBLE  | 5                                        |
		+------------+-----------------+------------------------------------------+
		| 40.2       | LITERAL_DOUBLE  | 12                                       |
		+------------+-----------------+------------------------------------------+
		| 987.654    | LITERAL_DOUBLE  | 14                                       |
		+------------+-----------------+------------------------------------------+
	\end{verbatim}
\end{flushleft}

\newpage

\subsection*{Caso de prueba 4: Formato Literales}
\addcontentsline{toc}{subsection}{Caso de prueba 4: Formato Literales}
\begin{flushleft}
	\begin{verbatim}
		#include <stdio.h>
		
		int main() {
			// Números decimales
			int decimal = 12345; // Decimal positivo
			int negativeDecimal = -6789; // Decimal negativo
			
			// Números octales
			int octal = 0123; // Octal (equivalente a 83 en decimal)
			int negativeOctal = -07654; // Octal negativo (equivalente a -4012 en decimal)
			
			// Números hexadecimales
			int hexadecimal = 0x1A3F; // Hexadecimal (equivalente a 6719 en decimal)
			int negativeHexadecimal = -0xDEAD; // Hexadecimal negativo (equivalente a -57005 en decimal)
			
			// Números binarios (C11 en adelante)
			int binary = 0b1101; // Binario (equivalente a 13 en decimal)
			
			// Números con punto flotante
			float floatNum = 3.14159; // Flotante positivo
			float negativeFloat = -0.9876; // Flotante negativo
			
			// Números en notación exponencial (notación científica)
			double scientificPos = 1.23e4; // 1.23 * 10^4 = 12300
			double scientificNeg = -5.67E-3; // -5.67 * 10^-3 = -0.00567
			
			// Números flotantes con notación hexadecimal (C99)
			double hexFloat = 0x1.1p3; // Equivalente a 8.5 en decimal
			
			// Strings-caracteres
			char *f = "Cadena válida";      // Cadena válida de caracteres
			char g = 'a';                   // Caracter válido
			char h = #65;                 // Literal char válido
			
			// Errores intencionales:
			int invalidHex = 0x1G; // Error: carácter inválido en número hexadecimal
			float invalidExp = 12.34e+; // Error: exponente no válido
			int invalidOctal = 089; // Error: dígitos no válidos en número octal
			double invalidFloat = 1.23.45; // Error: punto flotante mal formado
			    double invalid = 5..38; // Punto flotante mal formado
			int invalidInt = 123abc;         // Número entero inválido
			char *s = "Esto es un \n
			string inválido porque está en múltiples líneas";           
			
			return 0;
		}
	\end{verbatim}
	Errores esperados:
	\begin{itemize}
		\item Error en la línea 1: Operador inválido '\#'.
		\item Error en la línea 28: Formato de número no válido '.1'.
		\item Error en la línea 31: Dígito antes de id '0x1G'.
		\item Error en la línea 34: Formato de número no válido '.45'.
		\item Error en la línea 40: Formato de número no válido '1.23.45'.
		\item Error en la línea 41: Formato de número no válido '5..38'.
		\item Error en la línea 43: String no valido "Esto es un string inválido porque está en múltiples líneas".
	\end{itemize}
	Resultados:
	\begin{verbatim}
	Errors: 
	Character unknown: # in 1
	Invalid number format: .1 in 28
	Digit before id: 0x1G in 31
	Invalid number format: .45 in 34
	Invalid number format: 1.23.45 in 40
	Invalid number format: 5..38 in 41
	Strings cannot span multiple lines: "Esto es un \n string inválido porque está en múltiples líneas" in 43
	+------------+-----------------+------------------------------------------+
	| Token      | Tipo de Token   | Linea                                    |
	+------------+-----------------+------------------------------------------+
	| char       | KEYWORD         | 31, 32, 33                               |
	+------------+-----------------+------------------------------------------+
	| double     | KEYWORD         | 24, 25, 28, 40, 41                       |
	+------------+-----------------+------------------------------------------+
	| float      | KEYWORD         | 20, 21, 38                               |
	+------------+-----------------+------------------------------------------+
	| int        | KEYWORD         | 3, 5, 6, 9, 10, 13, 14, 17, 37, 39, 42   |
	+------------+-----------------+------------------------------------------+
	| return     | KEYWORD         | 44                                       |
	+------------+-----------------+------------------------------------------+
	| binary     | ID              | 17                                       |
	+------------+-----------------+------------------------------------------+
	| decimal    | ID              | 5                                        |
	+------------+-----------------+------------------------------------------+
	| e          | ID              | 38                                       |
	+------------+-----------------+------------------------------------------+
	| f          | ID              | 31                                       |
	+------------+-----------------+------------------------------------------+
	| floatNum   | ID              | 20                                       |
	+------------+-----------------+------------------------------------------+
	| g          | ID              | 32                                       |
	+------------+-----------------+------------------------------------------+
	| h          | ID              | 1, 33                                    |
	+------------+-----------------+------------------------------------------+
	| hexFloat   | ID              | 28                                       |
	+------------+-----------------+------------------------------------------+
	| hexadecima | ID              | 13                                       |
	| l          |                 |                                          |
	+------------+-----------------+------------------------------------------+
	| include    | ID              | 1                                        |
	+------------+-----------------+------------------------------------------+
	| invalid    | ID              | 41                                       |
	+------------+-----------------+------------------------------------------+
	| invalidExp | ID              | 38                                       |
	+------------+-----------------+------------------------------------------+
	| invalidFlo | ID              | 40                                       |
	| at         |                 |                                          |
	+------------+-----------------+------------------------------------------+
	| invalidHex | ID              | 37                                       |
	+------------+-----------------+------------------------------------------+
	| invalidInt | ID              | 42                                       |
	+------------+-----------------+------------------------------------------+
	| invalidOct | ID              | 39                                       |
	| al         |                 |                                          |
	+------------+-----------------+------------------------------------------+
	| main       | ID              | 3                                        |
	+------------+-----------------+------------------------------------------+
	| negativeDe | ID              | 6                                        |
	| cimal      |                 |                                          |
	+------------+-----------------+------------------------------------------+
	| negativeFl | ID              | 21                                       |
	| oat        |                 |                                          |
	+------------+-----------------+------------------------------------------+
	| negativeHe | ID              | 14                                       |
	| xadecimal  |                 |                                          |
	+------------+-----------------+------------------------------------------+
	| negativeOc | ID              | 10                                       |
	| tal        |                 |                                          |
	+------------+-----------------+------------------------------------------+
	| octal      | ID              | 9                                        |
	+------------+-----------------+------------------------------------------+
	| scientific | ID              | 25                                       |
	| Neg        |                 |                                          |
	+------------+-----------------+------------------------------------------+
	| scientific | ID              | 24                                       |
	| Pos        |                 |                                          |
	+------------+-----------------+------------------------------------------+
	| stdio      | ID              | 1                                        |
	+------------+-----------------+------------------------------------------+
	| xDEAD      | ID              | 14                                       |
	+------------+-----------------+------------------------------------------+
	| (          | OPERATOR        | 3                                        |
	+------------+-----------------+------------------------------------------+
	| )          | OPERATOR        | 3                                        |
	+------------+-----------------+------------------------------------------+
	| .          | OPERATOR        | 1, 28                                    |
	+------------+-----------------+------------------------------------------+
	| ;          | OPERATOR        | 5, 6, 9, 10, 13, 14, 17, 20, 21, 24, 25, |
	|            |                 |  28, 31, 32, 33, 37, 38, 39, 40, 41, 42, |
	|            |                 |  44                                      |
	+------------+-----------------+------------------------------------------+
	| =          | OPERATOR        | 5, 6, 9, 10, 13, 14, 17, 20, 21, 24, 25, |
	|            |                 |  28, 31, 32, 33, 37, 38, 39, 40, 41, 42  |
	+------------+-----------------+------------------------------------------+
	| {          | OPERATOR        | 3                                        |
	+------------+-----------------+------------------------------------------+
	| }          | OPERATOR        | 45                                       |
	+------------+-----------------+------------------------------------------+
	| *          | OPERATOR_ARITHM | 31                                       |
	|            | ETIC            |                                          |
	+------------+-----------------+------------------------------------------+
	| +          | OPERATOR_ARITHM | 38                                       |
	|            | ETIC            |                                          |
	+------------+-----------------+------------------------------------------+
	| <          | OPERATOR_RELATI | 1                                        |
	|            | ONAL            |                                          |
	+------------+-----------------+------------------------------------------+
	| >          | OPERATOR_RELATI | 1                                        |
	|            | ONAL            |                                          |
	+------------+-----------------+------------------------------------------+
	| -0         | LITERAL_INT     | 10, 14                                   |
	+------------+-----------------+------------------------------------------+
	| -6789      | LITERAL_INT     | 6                                        |
	+------------+-----------------+------------------------------------------+
	| 0          | LITERAL_INT     | 39, 44                                   |
	+------------+-----------------+------------------------------------------+
	| 12345      | LITERAL_INT     | 5                                        |
	+------------+-----------------+------------------------------------------+
	| 7654       | LITERAL_INT     | 10                                       |
	+------------+-----------------+------------------------------------------+
	| 89         | LITERAL_INT     | 39                                       |
	+------------+-----------------+------------------------------------------+
	| "Cadena vá | LITERAL_STR     | 31                                       |
	| lida"      |                 |                                          |
	+------------+-----------------+------------------------------------------+
	| #65        | LITERAL_CHAR    | 33                                       |
	+------------+-----------------+------------------------------------------+
	| 'a'        | LITERAL_CHAR    | 32                                       |
	+------------+-----------------+------------------------------------------+
	| 0b1101     | LITERAL_BINARY  | 17                                       |
	+------------+-----------------+------------------------------------------+
	| 0123       | LITERAL_OCTAL   | 9                                        |
	+------------+-----------------+------------------------------------------+
	| 0x1        | LITERAL_HEX     | 28                                       |
	+------------+-----------------+------------------------------------------+
	| 0x1A3F     | LITERAL_HEX     | 13                                       |
	+------------+-----------------+------------------------------------------+
	| -0.9876    | LITERAL_DOUBLE  | 21                                       |
	+------------+-----------------+------------------------------------------+
	| -5.67E-3   | LITERAL_DOUBLE  | 25                                       |
	+------------+-----------------+------------------------------------------+
	| 1.23e4     | LITERAL_DOUBLE  | 24                                       |
	+------------+-----------------+------------------------------------------+
	| 12.34      | LITERAL_DOUBLE  | 38                                       |
	+------------+-----------------+------------------------------------------+
	| 3.14159    | LITERAL_DOUBLE  | 20                                       |
	+------------+-----------------+------------------------------------------+
	\end{verbatim}
\end{flushleft}

\subsection*{Caso de prueba 5: Palabras reservadas}
\addcontentsline{toc}{subsection}{Caso de prueba 5: Palabras reservadas}
\begin{flushleft}
	\begin{verbatim}
		#include <stdio.h>
		
		int main() {
			auto intVar = 5;          // auto
			int b = intVar + 3;       // int
			char c = 'A';             // char
			const int d = 10;         // const
			unsgned long g = 100L;   
			signed shrt h = -10;      
			statc int counter = 0;    
			float e = 3.14f;          // float
			double f = 5.0;           // double
			unsigned long g = 100L;   // unsigned long
			signed short h = -10;     // signed y short
			enum days { Mon, Tue };   // enum
			struct point {            // struct
				int x, y;
			};
			typedef int entero;       // typedef'
			extern int i;             // extern'
			static int counter = 0;   // static'
			void function() {         // void
				continue;             // continue
			}
			
			
			return 0;
		}
	\end{verbatim}

	Resultados:
	\begin{verbatim}
		+------------+-----------------+------------------------------------------+
		| Token      | Tipo de Token   | Linea                                    |
		+------------+-----------------+------------------------------------------+
		| auto       | KEYWORD         | 2                                        |
		+------------+-----------------+------------------------------------------+
		| char       | KEYWORD         | 4                                        |
		+------------+-----------------+------------------------------------------+
		| const      | KEYWORD         | 5                                        |
		+------------+-----------------+------------------------------------------+
		| continue   | KEYWORD         | 21                                       |
		+------------+-----------------+------------------------------------------+
		| double     | KEYWORD         | 10                                       |
		+------------+-----------------+------------------------------------------+
		| enum       | KEYWORD         | 13                                       |
		+------------+-----------------+------------------------------------------+
		| extern     | KEYWORD         | 18                                       |
		+------------+-----------------+------------------------------------------+
		| float      | KEYWORD         | 9                                        |
		+------------+-----------------+------------------------------------------+
		| int        | KEYWORD         | 1, 3, 5, 8, 15, 17, 18, 19               |
		+------------+-----------------+------------------------------------------+
		| long       | KEYWORD         | 6, 11                                    |
		+------------+-----------------+------------------------------------------+
		| return     | KEYWORD         | 25                                       |
		+------------+-----------------+------------------------------------------+
		| short      | KEYWORD         | 12                                       |
		+------------+-----------------+------------------------------------------+
		| signed     | KEYWORD         | 7, 12                                    |
		+------------+-----------------+------------------------------------------+
		| static     | KEYWORD         | 19                                       |
		+------------+-----------------+------------------------------------------+
		| struct     | KEYWORD         | 14                                       |
		+------------+-----------------+------------------------------------------+
		| typedef    | KEYWORD         | 17                                       |
		+------------+-----------------+------------------------------------------+
		| unsigned   | KEYWORD         | 11                                       |
		+------------+-----------------+------------------------------------------+
		| void       | KEYWORD         | 20                                       |
		+------------+-----------------+------------------------------------------+
		| Mon        | ID              | 13                                       |
		+------------+-----------------+------------------------------------------+
		| Tue        | ID              | 13                                       |
		+------------+-----------------+------------------------------------------+
		| b          | ID              | 3                                        |
		+------------+-----------------+------------------------------------------+
		| c          | ID              | 4                                        |
		+------------+-----------------+------------------------------------------+
		| counter    | ID              | 8, 19                                    |
		+------------+-----------------+------------------------------------------+
		| d          | ID              | 5                                        |
		+------------+-----------------+------------------------------------------+
		| days       | ID              | 13                                       |
		+------------+-----------------+------------------------------------------+
		| e          | ID              | 9                                        |
		+------------+-----------------+------------------------------------------+
		| entero     | ID              | 17                                       |
		+------------+-----------------+------------------------------------------+
		| f          | ID              | 10                                       |
		+------------+-----------------+------------------------------------------+
		| function   | ID              | 20                                       |
		+------------+-----------------+------------------------------------------+
		| g          | ID              | 6, 11                                    |
		+------------+-----------------+------------------------------------------+
		| h          | ID              | 7, 12                                    |
		+------------+-----------------+------------------------------------------+
		| i          | ID              | 18                                       |
		+------------+-----------------+------------------------------------------+
		| intVar     | ID              | 2, 3                                     |
		+------------+-----------------+------------------------------------------+
		| main       | ID              | 1                                        |
		+------------+-----------------+------------------------------------------+
		| point      | ID              | 14                                       |
		+------------+-----------------+------------------------------------------+
		| shrt       | ID              | 7                                        |
		+------------+-----------------+------------------------------------------+
		| statc      | ID              | 8                                        |
		+------------+-----------------+------------------------------------------+
		| unsgned    | ID              | 6                                        |
		+------------+-----------------+------------------------------------------+
		| x          | ID              | 15                                       |
		+------------+-----------------+------------------------------------------+
		| y          | ID              | 15                                       |
		+------------+-----------------+------------------------------------------+
		| (          | OPERATOR        | 1, 20                                    |
		+------------+-----------------+------------------------------------------+
		| )          | OPERATOR        | 1, 20                                    |
		+------------+-----------------+------------------------------------------+
		| ,          | OPERATOR        | 13, 15                                   |
		+------------+-----------------+------------------------------------------+
		| ;          | OPERATOR        | 2, 3, 4, 5, 6, 7, 8, 9, 10, 11, 12, 13,  |
		|            |                 | 15, 16, 17, 18, 19, 21, 25               |
		+------------+-----------------+------------------------------------------+
		| =          | OPERATOR        | 2, 3, 4, 5, 6, 7, 8, 9, 10, 11, 12, 19   |
		+------------+-----------------+------------------------------------------+
		| {          | OPERATOR        | 1, 13, 14, 20                            |
		+------------+-----------------+------------------------------------------+
		| }          | OPERATOR        | 13, 16, 22, 26                           |
		+------------+-----------------+------------------------------------------+
		| +          | OPERATOR_ARITHM | 3                                        |
		|            | ETIC            |                                          |
		+------------+-----------------+------------------------------------------+
		| -10        | LITERAL_INT     | 7, 12                                    |
		+------------+-----------------+------------------------------------------+
		| 0          | LITERAL_INT     | 8, 19, 25                                |
		+------------+-----------------+------------------------------------------+
		| 10         | LITERAL_INT     | 5                                        |
		+------------+-----------------+------------------------------------------+
		| 100L       | LITERAL_INT     | 6, 11                                    |
		+------------+-----------------+------------------------------------------+
		| 3          | LITERAL_INT     | 3                                        |
		+------------+-----------------+------------------------------------------+
		| 5          | LITERAL_INT     | 2                                        |
		+------------+-----------------+------------------------------------------+
		| 'A'        | LITERAL_CHAR    | 4                                        |
		+------------+-----------------+------------------------------------------+
		| 3.14f      | LITERAL_DOUBLE  | 9                                        |
		+------------+-----------------+------------------------------------------+
		| 5.0        | LITERAL_DOUBLE  | 10                                       |
		+------------+-----------------+------------------------------------------+
	\end{verbatim}
\end{flushleft}
\subsection*{Caso de prueba 6: General}
\addcontentsline{toc}{subsection}{Caso de prueba 6: General}
\begin{flushleft}
	\begin{verbatim}
		#include <stdio.h>
		
		int main() {
			// Palabras reservadas
			int x = 10;                // int
			float y = 5.5f;           // float
			char z = 'c';             // char
			const int max = 100;      // const
			double pi = 3.14159;      // double
			unsigned int u = 5;       // unsigned
			signed int s = -3;        // signed
			enum colors { RED, GREEN, BLUE }; // enum
			struct Point {            // struct
				int x, y;
			};
			
			// Usando typedef
			typedef int entero;       // typedef
			entero a = 5;             // usando typedef
			
			// Operaciones
			int sum = x + 20;         // Suma
			int product = x * 2;      // Multiplicación
			double division = (double)x / 4; // División
			int modulo = x % 3;       // Módulo
			
			// Operadores bit a bit
			int andBit = x & 1;       // AND bit a bit
			int orBit = x | 1;        // OR bit a bit
			int xorBit = x ^ 1;       // XOR bit a bit
			int notBit = ~x;          // NOT bit a bit
			
			// Desplazamientos
			int leftShift = x << 1;   // Desplazamiento a la izquierda
			int rightShift = x >> 1;  // Desplazamiento a la derecha
			
			// Errores intencionales
			floaat wrongFloat = 2.5;   // Error: debería ser 'float'
			int 2ndVar = 10;           // Error: no puede comenzar con un dígito
			const float pi = 3.14;      // Error: redeclaración de 'const'
			int l = a @ b;          // Operador inválido
			int n = a // b;         // Comentario mal formado
			
			return 0;
		}
		
	\end{verbatim}
	Errores esperados:
	\begin{itemize}
		\item Error en la línea 38: Dígito antes de id '2ndVar'.
		\item Error en la línea 28: Caracter desconocido '@'.
	\end{itemize}
	Resultados:
	\begin{verbatim}
		Errors: 
		Digit before id: 2ndVar in 38
		Character unknown: @ in 40
		+------------+-----------------+------------------------------------------+
		| Token      | Tipo de Token   | Linea                                    |
		+------------+-----------------+------------------------------------------+
		| char       | KEYWORD         | 5, 32, 33, 34                            |
		+------------+-----------------+------------------------------------------+
		| const      | KEYWORD         | 6, 39                                    |
		+------------+-----------------+------------------------------------------+
		| double     | KEYWORD         | 7, 22(2)                                 |
		+------------+-----------------+------------------------------------------+
		| enum       | KEYWORD         | 10                                       |
		+------------+-----------------+------------------------------------------+
		| float      | KEYWORD         | 4, 39                                    |
		+------------+-----------------+------------------------------------------+
		| int        | KEYWORD         | 1, 3, 6, 8, 9, 12, 16, 20, 21, 23, 26, 2 |
		|            |                 | 7, 28, 29, 38, 40, 41                    |
		+------------+-----------------+------------------------------------------+
		| return     | KEYWORD         | 43                                       |
		+------------+-----------------+------------------------------------------+
		| signed     | KEYWORD         | 9                                        |
		+------------+-----------------+------------------------------------------+
		| struct     | KEYWORD         | 11                                       |
		+------------+-----------------+------------------------------------------+
		| typedef    | KEYWORD         | 16                                       |
		+------------+-----------------+------------------------------------------+
		| unsigned   | KEYWORD         | 8                                        |
		+------------+-----------------+------------------------------------------+
		| BLUE       | ID              | 10                                       |
		+------------+-----------------+------------------------------------------+
		| GREEN      | ID              | 10                                       |
		+------------+-----------------+------------------------------------------+
		| Point      | ID              | 11                                       |
		+------------+-----------------+------------------------------------------+
		| RED        | ID              | 10                                       |
		+------------+-----------------+------------------------------------------+
		| a          | ID              | 17, 40, 41                               |
		+------------+-----------------+------------------------------------------+
		| andBit     | ID              | 26                                       |
		+------------+-----------------+------------------------------------------+
		| b          | ID              | 40                                       |
		+------------+-----------------+------------------------------------------+
		| colors     | ID              | 10                                       |
		+------------+-----------------+------------------------------------------+
		| division   | ID              | 22                                       |
		+------------+-----------------+------------------------------------------+
		| entero     | ID              | 16, 17                                   |
		+------------+-----------------+------------------------------------------+
		| f          | ID              | 32                                       |
		+------------+-----------------+------------------------------------------+
		| floaat     | ID              | 37                                       |
		+------------+-----------------+------------------------------------------+
		| g          | ID              | 33                                       |
		+------------+-----------------+------------------------------------------+
		| h          | ID              | 34                                       |
		+------------+-----------------+------------------------------------------+
		| l          | ID              | 40                                       |
		+------------+-----------------+------------------------------------------+
		| main       | ID              | 1                                        |
		+------------+-----------------+------------------------------------------+
		| max        | ID              | 6                                        |
		+------------+-----------------+------------------------------------------+
		| modulo     | ID              | 23                                       |
		+------------+-----------------+------------------------------------------+
		| n          | ID              | 41                                       |
		+------------+-----------------+------------------------------------------+
		| notBit     | ID              | 29                                       |
		+------------+-----------------+------------------------------------------+
		| orBit      | ID              | 27                                       |
		+------------+-----------------+------------------------------------------+
		| pi         | ID              | 7, 39                                    |
		+------------+-----------------+------------------------------------------+
		| product    | ID              | 21                                       |
		+------------+-----------------+------------------------------------------+
		| s          | ID              | 9                                        |
		+------------+-----------------+------------------------------------------+
		| sum        | ID              | 20                                       |
		+------------+-----------------+------------------------------------------+
		| u          | ID              | 8                                        |
		+------------+-----------------+------------------------------------------+
		| wrongFloat | ID              | 37                                       |
		+------------+-----------------+------------------------------------------+
		| x          | ID              | 3, 12, 20, 21, 22, 23, 26, 27, 28, 29    |
		+------------+-----------------+------------------------------------------+
		| xorBit     | ID              | 28                                       |
		+------------+-----------------+------------------------------------------+
		| y          | ID              | 4, 12                                    |
		+------------+-----------------+------------------------------------------+
		| z          | ID              | 5                                        |
		+------------+-----------------+------------------------------------------+
		| (          | OPERATOR        | 1, 22                                    |
		+------------+-----------------+------------------------------------------+
		| )          | OPERATOR        | 1, 22                                    |
		+------------+-----------------+------------------------------------------+
		| ,          | OPERATOR        | 10(2), 12                                |
		+------------+-----------------+------------------------------------------+
		| ;          | OPERATOR        | 3, 4, 5, 6, 7, 8, 9, 10, 12, 13, 16, 17, |
		|            |                 |  20, 21, 22, 23, 26, 27, 28, 29, 32, 33, |
		|            |                 |  34, 37, 38, 39, 40, 43                  |
		+------------+-----------------+------------------------------------------+
		| =          | OPERATOR        | 3, 4, 5, 6, 7, 8, 9, 17, 20, 21, 22, 23, |
		|            |                 |  26, 27, 28, 29, 32, 33, 34, 37, 38, 39, |
		|            |                 |  40, 41                                  |
		+------------+-----------------+------------------------------------------+
		| {          | OPERATOR        | 1, 10, 11                                |
			+------------+-----------------+------------------------------------------+
			| }          | OPERATOR        | 10, 13, 44                               |
		+------------+-----------------+------------------------------------------+
		| %          | OPERATOR_ARITHM | 23                                       |
		|            | ETIC            |                                          |
		+------------+-----------------+------------------------------------------+
		| *          | OPERATOR_ARITHM | 21, 32                                   |
		|            | ETIC            |                                          |
		+------------+-----------------+------------------------------------------+
		| +          | OPERATOR_ARITHM | 20                                       |
		|            | ETIC            |                                          |
		+------------+-----------------+------------------------------------------+
		| /          | OPERATOR_ARITHM | 22                                       |
		|            | ETIC            |                                          |
		+------------+-----------------+------------------------------------------+
		| &          | OPERATOR_BITWIS | 26                                       |
		|            | E               |                                          |
		+------------+-----------------+------------------------------------------+
		| ^          | OPERATOR_BITWIS | 28                                       |
		|            | E               |                                          |
		+------------+-----------------+------------------------------------------+
		| |          | OPERATOR_BITWIS | 27                                       |
		|            | E               |                                          |
		+------------+-----------------+------------------------------------------+
		| ~          | OPERATOR_BITWIS | 29                                       |
		|            | E               |                                          |
		+------------+-----------------+------------------------------------------+
		| -3         | LITERAL_INT     | 9                                        |
		+------------+-----------------+------------------------------------------+
		| 0          | LITERAL_INT     | 43                                       |
		+------------+-----------------+------------------------------------------+
		| 1          | LITERAL_INT     | 26, 27, 28                               |
		+------------+-----------------+------------------------------------------+
		| 10         | LITERAL_INT     | 3, 38                                    |
		+------------+-----------------+------------------------------------------+
		| 100        | LITERAL_INT     | 6                                        |
		+------------+-----------------+------------------------------------------+
		| 2          | LITERAL_INT     | 21                                       |
		+------------+-----------------+------------------------------------------+
		| 20         | LITERAL_INT     | 20                                       |
		+------------+-----------------+------------------------------------------+
		| 3          | LITERAL_INT     | 23                                       |
		+------------+-----------------+------------------------------------------+
		| 4          | LITERAL_INT     | 22                                       |
		+------------+-----------------+------------------------------------------+
		| 5          | LITERAL_INT     | 8, 17                                    |
		+------------+-----------------+------------------------------------------+
		| "Cadena vá | LITERAL_STR     | 32                                       |
		| lida"      |                 |                                          |
		+------------+-----------------+------------------------------------------+
		| #65        | LITERAL_CHAR    | 34                                       |
		+------------+-----------------+------------------------------------------+
		| 'a'        | LITERAL_CHAR    | 33                                       |
		+------------+-----------------+------------------------------------------+
		| 'c'        | LITERAL_CHAR    | 5                                        |
		+------------+-----------------+------------------------------------------+
		| 2.5        | LITERAL_DOUBLE  | 37                                       |
		+------------+-----------------+------------------------------------------+
		| 3.14       | LITERAL_DOUBLE  | 39                                       |
		+------------+-----------------+------------------------------------------+
		| 3.14159    | LITERAL_DOUBLE  | 7                                        |
		+------------+-----------------+------------------------------------------+
		| 5.5f       | LITERAL_DOUBLE  | 4                                        |
	\end{verbatim}
\end{flushleft}


% Manual de usuario
\section*{Manual de usuario}
\addcontentsline{toc}{section}{Manual de usuario}
\subsection*{Instalación}
Para construir y ejecutar el proyecto, es necesario tener Java instalado en tu sistema. Sigue estos pasos para configurar el proyecto:

\begin{enumerate}
    \item Clona el repositorio:
    \begin{verbatim}
    git clone https://github.com/AlonsoNav/CCompilerJFlex.git
    cd your-repo
    \end{verbatim}

    \item Genera el archivo \texttt{CLexer}:
    \begin{verbatim}
    java -jar lib/jflex-full-1.9.1.jar src/scanner/CLexer.flex
    \end{verbatim}

	\item Genera el archivo \texttt{Parser}:
	\begin{verbatim}
	java -jar lib/java-cup-11b.jar -parser Parser -symbols Symbol 
	src/parser/Parser.cup
	\end{verbatim}

    \item Compila el proyecto:
    \begin{verbatim}
    javac -d bin -sourcepath src -cp lib/java-cup-11b.jar 
	src/app/ParserMain.java src/app/ScannerMain.java 
	src/scanner/CLexer.java src/scanner/Token.java 
	src/scanner/TokenType.java src/parser/Parser.java 
	src/parser/Sym.java
    \end{verbatim}
\end{enumerate}

\subsection*{Uso}
Para ejecutar el compilador con un archivo de entrada, utiliza el siguiente comando:
\begin{verbatim}
    java -cp bin app.ParserMain input_file
\end{verbatim}

% Bitácora
\section*{Bitácora}
\addcontentsline{toc}{section}{Bitácora}

\subsection*{Fecha: 17-10-2024}
\begin{flushleft}
    \hspace*{2em} Para iniciar el proyecto hicimos una pequeña
    reunión en la que se estableció que CV se encargaba de las
    variables, declaraciones y constantes. Por otro lado, VG
	iba a trabajar las estructuras de control como condicionales
	y bucles. AN diseñaría las gramáticas para las expresiones, 
	y las funciones como read y write. CV también trabajó en la 
	conversión de los tokens del scanners a Symbol y además de dar 
	la estructura básica del programa. A continuación se adjunta 
	evidencia de lo hablado en un chat de WhatsApp:\par\vspace{1em}
	\centering
	\includegraphics[width=0.5\textwidth]{log_1.jpg}\par
\end{flushleft}

\subsection*{Fecha: 26-10-2024}
\begin{flushleft}
    En este día nos conectamos un rato en Discord para revisar 
	lo que tenemos trabajado hasta el momento y discutir las 
	próximas etapas del proyecto. Para el manejo de errores se hizo 
	otra distribución donde CV se encargaba de errores de expresiones 
	y de read and write, AN de errores de estructuras de control y VG 
	de errores de variables, constantes y funciones.
\end{flushleft}

\subsection*{Fecha: 04-09-2024}
\begin{flushleft}
    \hspace*{2em} Para este punto se tienen muchas complicaciones 
	con el manejo de errores, por lo que a partir de este día se 
	tienen reuniones consecutivas donde se hace trabajo cooperativo 
	con el fin de sacar las funcionalidades. Por último, se divide 
	las partes de la documentación que serán redactadas por todos 
	los miembros del equipo.
\end{flushleft}

% Bibliografía
\newpage
\section*{Bibliografía}
\addcontentsline{toc}{section}{Bibliografía}
\begin{thebibliography}{9}

\bibitem{examplewebsite}
Klein, G., Rowe, S., \& Décamps, R. (marzo de 2023).
\emph{JFlex User’s Manual}.
JFlex Team.
En: \url{https://www.jflex.de/manual.html}.

\bibitem{examplewebsite}
Hudson, S. (julio de 1999).
\emph{CUP User’s Manual}.
Cup.
En: \url{https://www.cs.princeton.edu/~appel/modern/java/CUP/manual.html#intro}.

\end{thebibliography}

\end{document}